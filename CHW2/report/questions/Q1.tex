\section{کاهش نویز در سیگنال گرافی}

کد ها در فایل های \texttt{Q1.m} و \texttt{Q2.m} ضمیمه شده اند، این گزارشی از نتایج شبیه سازی ها و
محسابات خواهد بود.

\subsection{تولید و تعریف گراف}

در حل این مسئله مکان راس ها اهمیتی ندارد، در نتیجه برای مکان این راس ها مکان های انتزاعی با استفاده از \texttt{sensor graph}
تخصیص داده ایم.
\begin{figure}[h!]
	\centering
	\includegraphics*[width=0.55\linewidth]{../results/Q1/graph.png}
	\vspace*{-3em}
	\caption{رسم گراف}
\end{figure}

\subsection{تعریف سیگنال اصلی و با نویز}

با استفاده از طیف لاپلاسی کراف، سیگنال اصلی $x=2u_1+u_2$ و سیگنال نویزی $x_n=x+w$ را تعریف میکنیم، در شکل زیر این
سیگنال ها را روی گراف نمایش میدهیم.

\begin{figure}[h!]
	\centering
	\includegraphics*[width=0.9\linewidth]{../results/Q1/sig_noisysig.png}
	\vspace*{-2em}
	\caption{رسم سیگنال اصلی و نویزی بر روی گراف و به صورت سری زمانی}
\end{figure}

\subsection{طیف سیگنال}

حال ماتریس لاپلاسی و وزن نورمالیزه که به صورت $W_n=D^{-1/2}WD^{-1/2}$ تعریف میشنوند را تشکیل میدهیم، و سپس
با قرار دادن آنها به عنوان اپراتور شیفت گرافی، طیف گراف را با استخراج سیگنال های ویژه این اپراتور ها و
تجزیه سیگنال خود به این پایه ها تشکیل داده و نمایش میدهیم.

\begin{figure}[h!]
	\centering
	\includegraphics*[width=0.9\linewidth]{../results/Q1/signal_spectra.png}
	\vspace*{-2em}
	\label{fig:sig_spectra}
	\caption{
		طیف سیگنال، فرکانس: محور افقی، نویزی و معمولی(جپ و راست)، لاپلاسی یا وزن نورمالیزه(بالا، پایین)
	}
\end{figure}

\subsection{ساخت فیلتر}

همانطور که از شکل \ref*{fig:sig_spectra} معلوم است، نگه داشتن 2 مقدار ویژه کوچک تر و بطور متقابل 2
مقدار ویژه بزرگ تر در شیفت با وزن نورمالیزه، باید فیلتر خوبی برای این سیستم باشد، متناظر با 2 فرکانس کم و موجود در سیگنال اصلی. به طور معادل:
\[
	h_L(\lambda)=\left[1,1,0,\dots,0\right]^T,\qquad h_{W_n}(\lambda)=\left[0,\dots,0,1,1\right]^T
\]

\begin{figure}[h!]
	\centering
	\includegraphics*[width=0.9\linewidth]{../results/Q1/ideal_filt.png}
	\vspace*{-1em}
	\label{fig:ideal_filt}
	\caption{
		فیلتر ایده آل، لاپلاسین و وزن نورمالیزه(چپ و راست)
	}
\end{figure}

که برای پاسخ فرکانس در واقغ $h(\lambda)$ را در فرکانس های موجود در گراف رسم کرده ایم.

\clearpage
\subsection{فیلتر کردن}

حال فقط سیگنال اصلی را با استفاده از فیلتر خود، فیلتر میکنیم به صورتی که:
\[
	x_{\text{filtered}}=U(\hat{x}\odot h),\qquad \hat{x}=U^Hx
\]

\begin{figure}[h]
	\centering
	\includegraphics*[width=\linewidth]{../results/Q1/sig_filtsig.png}
	\vspace*{-2em}
	\caption{از چپ به راست: سیگنال اصلی، سیگنال نویزی، فیلتر شده با فیلتر لاپلاسی، فیلتر شده با فیلتر وزن نورمالیزه}
	\label{fig:filtered_sig_ideal}
\end{figure}

همانطور که از شکل \ref*{fig:filtered_sig_ideal} پیداست، سیگنال های فیلتر شده دارای SNR بهبود یافته اند، و
همچنین که بهترین نتیجه برای فیلتر کردن با استفاده از فیلتر با شیفت لاپلاسین گرافی است، زیرا
سیگنال اصلی روی طیف این اپراتور تعریف شده است و در این حوزه هموار تر است.

\[
	\text{SNR}_{\text{default}}=11.29\longrightarrow \underline{\text{SNR}_{L}=28.4},\quad \text{SNR}_{W_n}=19.72
\]

\subsection{سنتز با FIR}

میخواهیم که پاسخ این 2 فیلتر تا حد ممکن مشابه هم باشد، از معیار $l_2$ برای این خطا استفاده میکنیم.
\begin{gather*}
	|x_{ideal}-x_{fir}|^2=|\hat{x}_{ideal}-\hat{x}_{fir}|^2=\hat{x}^2|h_{ideal}-h_{fir}|^2\Rightarrow h_{fir,opt}=\underset{h\in\text{FIR}}{\mathrm{argmin}}|h_{ideal}-h|^2
\end{gather*}
ادامه محسابات در صفحه بعد.
\begin{figure}[h]
	\centering
	\includegraphics*[width=\linewidth]{../results/Q1/FIR_filt.png}
	\vspace*{-1em}
	\caption{
		فیلتر FIR در مقابل ایده آل، لاپلاسین و وزن نورمالیزه(چپ و راست)
	}
	\label{fig:fir_filt}
\end{figure}

\clearpage

\begin{gather*}
	h_{FIR}(\lambda)=h_0+h_1\lambda+h_2\lambda^2,\quad \lambda=\left[\lambda_1,\lambda_2,\dots,\lambda_n\right]^T\\
	h_{FIR} =
	\underbrace{\begin{pmatrix}
			\lambda^0 & \lambda^1 & \lambda^2
		\end{pmatrix}}_{\Lambda_3}\begin{pmatrix}
		h_0 \\h_1\\h_2
	\end{pmatrix}\\
	h_{opt}=\underset{h\in\text{FIR}}{\mathrm{argmin}}|h_{ideal}-h|^2=\Lambda_3\underset{x}{\mathrm{argmin}}|h_{ideal}-\Lambda_3x|^2=\Lambda_3\Lambda_3^\dagger h_{ideal}\\
	\left(h_0,h_1,h_2\right)^T=\Lambda_3^\dagger h_{ideal}\\
	h_{opt}:\text{\lr{projecting $h_{ideal}$ onto $\mathbb{C}(\Lambda_3)$}}
\end{gather*}

پس طبق روابط بالا میتوانیم فیلتر FIR با طول دلخواه را پیدا کنیم که کمترین فاصله با فیلتر ایده آل را بسازد، در شکل \ref{fig:fir_filt} این فیلتر با طول 3 را نمایش داده ایم،
برای این فیلتر:
\[
	h_{L,0}=1.1685,\quad h_{L,1}=-0.4237,\quad h_{L,2}=0.0370\quad h_{W_n,0}=0.1568,\quad h_{W_n,1}=0.6164,\quad h_{W_n,2}=0.3743
\]

\subsection{فیلتر کردن با FIR}
حال با فیلتر هایی که در قسمت قبل ساختیم، فیلتر میکنیم.
\begin{figure}[h]
	\centering
	\includegraphics*[width=\linewidth]{../results/Q1/FIR-filtering.png}
	\vspace*{-2em}
	\caption{از چپ به راست: سیگنال اصلی، سیگنال نویزی، فیلتر شده با فیلتر لاپلاسی، فیلتر شده با فیلتر وزن نورمالیزه}
	\label{fig:filtered_sig_FIR}
\end{figure}

همانطور که پیداست، SNR در شکل \ref*{fig:filtered_sig_ideal} بسیار بهتر از \ref*{fig:filtered_sig_FIR} که انتظار میرفت،
با اقزایش طول فیلتر FIR ولی این فاصله کمتر و کمتر میشود.

\[
	\text{SNR}_{\text{default}}=7.635\longrightarrow \underline{\text{SNR}_{ّFIR,L}=13.48},\quad \text{SNR}_{FIR,W_n}=12.33
\]