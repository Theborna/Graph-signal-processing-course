\section{جداسازی سیگنال گرافی}

کد ها و نتایج شبیه سازی ضمیمه شده اند و در گزارش تکرار نشده اند.

\subsection{سیگنال نرم تصادفی}

ابتدا تعدادی گراف به صورت گراف سنسور تولید میکنیم، و با تحلیل های مناسب روی آنها، سیگنال های تصادفی ولی نرم بر روی این گراف ها تولید میکنیم.

\begin{minipage}{0.6\linewidth}
	حال کافیست از هر کدام از این گراف ها، ماتریس لاپلاسین مربوطه $L_i$ را استخراج کرده، و سیگنال ادقام شده از سیگنال های نرم
	$x_i$ قبل که به صورت $x=\sum_{k=1}^{K}x_k$ را با استفاده از مسئله بهینه سازی زیر، جداسازی کنیم.

	\[
		\min_{x_1,\dots,x_k}\sum_{i=1}^{K} x_i^TL_ix_i\quad
		s.t.\begin{cases}
			x=\sum_{i=1}^{K}x_i,   \\
			1^Tx_i=0 & i=1,\dots,k
		\end{cases}
	\]

	این بهینه سازی، جداسازی خوبی انجام میدهد، زیرا هر کدام از $x_i$ ها تنها بر روی $L_i$ هموار هستند، در نتیجه میدانیم برای پاسخ اصلی، این تابع هدف مقدار کوچکی دارد.
	حال به هر روشی که این مسئله را حل کنیم، به پاسخ مطلوبی میرسیم.

	در این گزارش، ما با استفاده از \texttt{CVX} مسئله را حل کرده ایم.
\end{minipage}
\begin{minipage}{0.4\linewidth}
	\centering
	\includegraphics*[width=\linewidth]{../results/sample_sensor_graph.png}
	\captionof{figure}{نمونه گراف سنسور}
\end{minipage}
\vspace*{1em}

پس از انجام تحلیل ها، از معیار SNR برای سنجش عمل کرد استفاده کرده ایم، مقادیر برای یک آزمایش به صورت زیر بوده.


\begin{minipage}{0.6\linewidth}
	\vspace*{1em}
	\centering
	\begin{latin}
		\begin{tabular}{c|c|c|c|c}
			\textbf{$\text{SNR}_1$} & \textbf{$\text{SNR}_2$} & \textbf{$\text{SNR}_3$} & \textbf{$\text{SNR}_4$} & \textbf{$\text{SNR}_{avg}$} \\
			\hline
			30.21                   & 21.31                   & 11.26                   & 17.77                   & 20.14
		\end{tabular}
		\captionof{table}{SNR Values (db)}
	\end{latin}
\end{minipage}
\begin{minipage}{0.4\linewidth}
	\begin{gather*}
		\text{SNR}_i = 10\log\left(\frac{||x_i||^2_2}{||x_i|-\tilde{x}_i|^2_2}\right)\\
		\text{SNR}_{avg} = \frac{1}{K}\sum_{i=1}^{K}\text{SNR}_i
	\end{gather*}
\end{minipage}
\vspace*{1em}

برای اطمینان بیشتر روی مقادیر و کاهش واریانس اندازه گیری، آزمایش را 20 بار انجام داده، و میانگین را گزارش میکنیم. سپس مرحله قبل را برای ابعاد مختلف سیگنال نیز انجام میدهیم.

مقادیر به شکل زیر بوده اند.

\vfil

\begin{latin}
	\centering
	\begin{tabular}{c|c|c|c}
		\textbf{N}   & 100   & 200   & 300   \\
		\hline
		\textbf{SNR} & 14.65 & 15.96 & 17.58 \\
	\end{tabular}
	\captionof{table}{SNR Values (db) vs dimentionality}
\end{latin}

\clearpage

\begin{figure}[h!]
	\centering
	\includegraphics*[width=0.8\linewidth]{../results/SNR_Part6.png}
\end{figure}

\subsection{جداسازی تصویر}

ابتدا 2 تصویر داده شده را در محیط متلب لوود کرده و به فرمت مورد نظر برای تحلیل های خود میاوریم، سپس سیگنال ناشی از ادقام این 2 تصویر را میسازیم.

\vfil

\begin{figure}[h]
	\centering
	\includegraphics*[width=\linewidth]{../results/faces.png}
	\caption{تصاویر و تصویر ادقام شده}
\end{figure}

حال نیاز داریم مراحل قبل را باری دیگر طی کنیم، با این تقاوت که در قبل، گراف ها معلوم بوده و سپس سیگنال های تصادفی همواری روی گراف های
دانسته شده تعریف کردیم.

در اینجا تصاویر معلوم اند، در نتیجه برای اعمال کار مشابه نیاز داریم گرافی تولید کنیم که سیگنال ها بر روی آن هموار باشند.

\clearpage

گراف خود را یک گراف $king's$ در نظر میگیریم، که اینگونه شباهت های مکانی را در گراف خود میبینیم، و برای خاص کردن گراف برای سیگنال خود،
وزن ها را به صورت زیر قرار میدهیم.

\[
	w_{k, ij}=\frac{1}{|(x_k)_i-(x_k)_j|+\epsilon},\quad (i, j)\in\mathcal{E}(\text{kings}),\quad 0<\epsilon<1
\]

در اینجا از $\epsilon=0.001$ استفاده کردیم.

برای اثبات ادعا خود روی همواری سیگنال ها روی گراف مربوطه، 500 مولفه فرکانسی اول سیگنال ها را روی گراف خود رسم میکنیم.

\begin{figure}[h]
	\centering
	\includegraphics*[width=0.9\linewidth]{../results/gft_smoothness.png}
	\caption{تبدیل فوریه گرافی GFT با ماتریس لاپلاسین به عنوان GSO}
\end{figure}

\subsection{جداسازی}

حال با مراحل مشابهی، جداسازی را انجام میدهیم، همانطور که معلوم است، این جداسازی به صورت خیلی خوبی انجام شده و عکس ها
به نحو خوبی جدا شده اند.

\begin{figure}[h!]
	\centering
	\includegraphics*[width=\linewidth]{../results/bss_faces.png}
	\vspace*{-4em}
	\caption{بازسازی و جداسازی روی تصاویر}
    \vspace*{-1em}
	\[
		\text{SNR}_1 = 15.34dB,\quad \text{SNR}_2=15.92dB
	\]
\end{figure}