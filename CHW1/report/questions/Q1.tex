\section{آشنایی با \textsc{GSPBOX}}

ابتدا این تولباکس را نصب کرده، و سپس کار های خواسته شده را انجام میدهیم.
کد کامل این بخش در فایل \texttt{Q1.m} ضمیمه شده است.
\subsection{تعریف و رسم گراف}

گراف های خواسته شده را به صورت دستی تعریف کرده، و رسم میکنیم،
مختصات را با استفاده از گراف های آماده GSPBOX میدهیم، در واقع برای گراف دوم از ring استفاده کرده
و برای گراف اول از comet استفاده میکنیم تا مختصات مد نظر ما را بدهد.

\begin{figure}[h]
	\centering
	\includegraphics*[width=0.7\linewidth]{../results/Q1/graphs.png}
	\caption{رسم گراف های $G_1$, $G_2$}
\end{figure}

\subsection{ضرب گراف}

با استفاده از تابغ \texttt{gsp\_graph\_product} ضرب های گرافی را تشکیل میدهیم.

\begin{figure}[h]
	\centering
	\includegraphics*[width=0.7\linewidth]{../results/Q1/graph_products.png}
	\caption{رسم گراف های $G_s$, $G_t$}
	\label{fig:products}
\end{figure}

همانطور که از \ref*{fig:products} معلوم است، نظم خاصی در گراف های تشکیل شده از راه ضرب گرافی وجود دارد.

برای ماتریس وزن و اتصالات گراف های تشکیل شده، نتابج در فایل های \texttt{Gs\_matrices.txt} ذخیره شده است. ولی در زیر آورده شده اند.
{
\vfil
\small
\begin{gather*}
	A(G_s)=\left(\begin{array}{cccccccccccc} 0 & 0 & 0 & 0 & 1 & 1 & 0 & 1 & 1 & 0 & 1 & 1\\ 0 & 0 & 0 & 1 & 0 & 1 & 1 & 0 & 1 & 1 & 0 & 1\\ 0 & 0 & 0 & 1 & 1 & 0 & 1 & 1 & 0 & 1 & 1 & 0\\ 0 & 1 & 1 & 0 & 0 & 0 & 0 & 0 & 0 & 0 & 0 & 0\\ 1 & 0 & 1 & 0 & 0 & 0 & 0 & 0 & 0 & 0 & 0 & 0\\ 1 & 1 & 0 & 0 & 0 & 0 & 0 & 0 & 0 & 0 & 0 & 0\\ 0 & 1 & 1 & 0 & 0 & 0 & 0 & 0 & 0 & 0 & 0 & 0\\ 1 & 0 & 1 & 0 & 0 & 0 & 0 & 0 & 0 & 0 & 0 & 0\\ 1 & 1 & 0 & 0 & 0 & 0 & 0 & 0 & 0 & 0 & 0 & 0\\ 0 & 1 & 1 & 0 & 0 & 0 & 0 & 0 & 0 & 0 & 0 & 0\\ 1 & 0 & 1 & 0 & 0 & 0 & 0 & 0 & 0 & 0 & 0 & 0\\ 1 & 1 & 0 & 0 & 0 & 0 & 0 & 0 & 0 & 0 & 0 & 0 \end{array}\right)\\
	W(G_s)=\left(\begin{array}{cccccccccccc} 0 & 0 & 0 & 0 & \frac{28}{25} & \frac{42}{25} & 0 & \frac{44}{25} & \frac{66}{25} & 0 & \frac{92}{25} & \frac{138}{25}\\ 0 & 0 & 0 & \frac{28}{25} & 0 & \frac{14}{25} & \frac{44}{25} & 0 & \frac{22}{25} & \frac{92}{25} & 0 & \frac{46}{25}\\ 0 & 0 & 0 & \frac{42}{25} & \frac{14}{25} & 0 & \frac{66}{25} & \frac{22}{25} & 0 & \frac{138}{25} & \frac{46}{25} & 0\\ 0 & \frac{28}{25} & \frac{42}{25} & 0 & 0 & 0 & 0 & 0 & 0 & 0 & 0 & 0\\ \frac{28}{25} & 0 & \frac{14}{25} & 0 & 0 & 0 & 0 & 0 & 0 & 0 & 0 & 0\\ \frac{42}{25} & \frac{14}{25} & 0 & 0 & 0 & 0 & 0 & 0 & 0 & 0 & 0 & 0\\ 0 & \frac{44}{25} & \frac{66}{25} & 0 & 0 & 0 & 0 & 0 & 0 & 0 & 0 & 0\\ \frac{44}{25} & 0 & \frac{22}{25} & 0 & 0 & 0 & 0 & 0 & 0 & 0 & 0 & 0\\ \frac{66}{25} & \frac{22}{25} & 0 & 0 & 0 & 0 & 0 & 0 & 0 & 0 & 0 & 0\\ 0 & \frac{92}{25} & \frac{138}{25} & 0 & 0 & 0 & 0 & 0 & 0 & 0 & 0 & 0\\ \frac{92}{25} & 0 & \frac{46}{25} & 0 & 0 & 0 & 0 & 0 & 0 & 0 & 0 & 0\\ \frac{138}{25} & \frac{46}{25} & 0 & 0 & 0 & 0 & 0 & 0 & 0 & 0 & 0 & 0 \end{array}\right)\\
	A(G_t)=\left(\begin{array}{cccccccccccc} 0 & 1 & 1 & 1 & 0 & 0 & 1 & 0 & 0 & 1 & 0 & 0\\ 1 & 0 & 1 & 0 & 1 & 0 & 0 & 1 & 0 & 0 & 1 & 0\\ 1 & 1 & 0 & 0 & 0 & 1 & 0 & 0 & 1 & 0 & 0 & 1\\ 1 & 0 & 0 & 0 & 1 & 1 & 0 & 0 & 0 & 0 & 0 & 0\\ 0 & 1 & 0 & 1 & 0 & 1 & 0 & 0 & 0 & 0 & 0 & 0\\ 0 & 0 & 1 & 1 & 1 & 0 & 0 & 0 & 0 & 0 & 0 & 0\\ 1 & 0 & 0 & 0 & 0 & 0 & 0 & 1 & 1 & 0 & 0 & 0\\ 0 & 1 & 0 & 0 & 0 & 0 & 1 & 0 & 1 & 0 & 0 & 0\\ 0 & 0 & 1 & 0 & 0 & 0 & 1 & 1 & 0 & 0 & 0 & 0\\ 1 & 0 & 0 & 0 & 0 & 0 & 0 & 0 & 0 & 0 & 1 & 1\\ 0 & 1 & 0 & 0 & 0 & 0 & 0 & 0 & 0 & 1 & 0 & 1\\ 0 & 0 & 1 & 0 & 0 & 0 & 0 & 0 & 0 & 1 & 1 & 0 \end{array}\right) \\
	W(G_t)=\left(\begin{array}{cccccccccccc} 0 & \frac{8}{5} & \frac{12}{5} & \frac{7}{10} & 0 & 0 & \frac{11}{10} & 0 & 0 & \frac{23}{10} & 0 & 0\\ \frac{8}{5} & 0 & \frac{4}{5} & 0 & \frac{7}{10} & 0 & 0 & \frac{11}{10} & 0 & 0 & \frac{23}{10} & 0\\ \frac{12}{5} & \frac{4}{5} & 0 & 0 & 0 & \frac{7}{10} & 0 & 0 & \frac{11}{10} & 0 & 0 & \frac{23}{10}\\ \frac{7}{10} & 0 & 0 & 0 & \frac{8}{5} & \frac{12}{5} & 0 & 0 & 0 & 0 & 0 & 0\\ 0 & \frac{7}{10} & 0 & \frac{8}{5} & 0 & \frac{4}{5} & 0 & 0 & 0 & 0 & 0 & 0\\ 0 & 0 & \frac{7}{10} & \frac{12}{5} & \frac{4}{5} & 0 & 0 & 0 & 0 & 0 & 0 & 0\\ \frac{11}{10} & 0 & 0 & 0 & 0 & 0 & 0 & \frac{8}{5} & \frac{12}{5} & 0 & 0 & 0\\ 0 & \frac{11}{10} & 0 & 0 & 0 & 0 & \frac{8}{5} & 0 & \frac{4}{5} & 0 & 0 & 0\\ 0 & 0 & \frac{11}{10} & 0 & 0 & 0 & \frac{12}{5} & \frac{4}{5} & 0 & 0 & 0 & 0\\ \frac{23}{10} & 0 & 0 & 0 & 0 & 0 & 0 & 0 & 0 & 0 & \frac{8}{5} & \frac{12}{5}\\ 0 & \frac{23}{10} & 0 & 0 & 0 & 0 & 0 & 0 & 0 & \frac{8}{5} & 0 & \frac{4}{5}\\ 0 & 0 & \frac{23}{10} & 0 & 0 & 0 & 0 & 0 & 0 & \frac{12}{5} & \frac{4}{5} & 0 \end{array}\right)
\end{gather*}
}

\clearpage

\subsection{بررسی فرکانسی}

برای ماتریس $H$ ما $G_t$ را انتخاب کرده ایم.
سیگنال رندم تولید شده به شکل زیر در آمده:

\begin{figure}[h]
	\centering
	\includegraphics*[width=0.5\linewidth]{../results/Q1/random_sig.png}
	\caption{نمایش سیگنال تصادفی}
	\label{fig:signal}
\end{figure}


\subsection{طیف گراف}

برای بررسی طیف گراف، تبدیل فوریه آن را گرفته و مقادیر ویژه را بررسی میکنیم، نمایش آن را با
یکی تبدیل فوریه گرفتن روی سیگنال دیده شده در شکل \ref*{fig:signal} انجام میدهیم و یکی صرفا با کشیدن
مقادیر ویژه.

\begin{figure}[h!]
	\centering
	\includegraphics*[width=0.6\linewidth]{../results/Q1/spectrum.png}
	\includegraphics*[width=0.6\linewidth]{../results/Q1/sig_frequencies.png}
	\caption{طیف گراف}
	\label{fig:spectrum}
\end{figure}

\subsection{سیگنال ویژه های گراف}

رسم بدین صورت در می آید.

\begin{figure}[h]
	\centering
	\includegraphics*[width=\linewidth]{../results/Q1/eigenvalues.png}
	\caption{سیگنال ویژه های گراف}
	\label{fig:eigvec}
\end{figure}

همانطور که از شکل \ref{fig:eigvec} واضح است، بردار ویژه های اول سیگنال های
هموار تری هستند، بدین صورت که اولین سیگنال ویؤه به کل ثابت بوده و دومین سیگنال ویژه در همسایگی
خود نسبتا ثابت است، آخرین بردار ویژه ها همواری خاصی ندارند.

\clearpage

\section{تشخیص گروه بندی در گراف و رسم گراف}

کد های مربوطه در فایل \texttt{Q2.m} ضمیمه شده اند.

شبکه دوستی را همانطور که گفته شده است، تشکیل میدهیم و گراف مد نظر را ساخته و تحلیل های فرکانسی مورد نظر را
روی گراف SBM ساخته شده انجام میدهیم.

در این تحلیل گفته شده که از ماتریس لاپلاسی گراف استفاده کنیم، ولی در عمل دیده شد که استغاده از ماتریس
لاپلاسی نورمالیزه به ما طبقه بنده ای میدهد که برای بسیاری طبقه بند ها مانند kmeans راحت تر و با دقت بالا نر است.

حال ما جفت این تحلیل ها را انجام داده ایم.

\begin{figure}[h]
	\centering
	\includegraphics*[width=0.49\linewidth]{../results/Q2/eigenmap_default.png}
	\includegraphics*[width=0.49\linewidth]{../results/Q2/eigenmap_norm.png}
	\caption{\lr{Laplacien eigenmaps} چپ(لاپلاسین نورمالیزه)، راست(لاپلاسین)}
	\label{fig:eigenmap}
\end{figure}

\subsection{حدود طبقه بندی درست}

از شکل \ref*{fig:eigenmap} واضح میشود که علامت بردار ویژه اول، میتواند معیار خوبی برای طبقه بندی باشد،
میتوانیم از الگوریتم های clustering متفاوت مانند kmeans نیز استفاده کنیم، در عمل همه این ها را
امتحان کرده ولی در گزارش نتیجه kmeans برای نولمالیزه و علامت برای لاپلاسین را آورده ایم.

\begin{figure}[h]
	\centering
	\includegraphics*[width=0.49\linewidth]{../results/Q2/acc_def_simple.png}
	\includegraphics*[width=0.49\linewidth]{../results/Q2/acc_norm.png}
	\caption{چپ(لاپلاسین نورمالیزه+kmeans)، راست(لاپلاسین+علامت)}
	\label{fig:clustering}
\end{figure}

همانطور که از شکل \ref*{fig:clustering} واضح است، در جفت روش ها ما موفق به رسیدن به دقت 
$100\%$ بعد از مقدار مشخصی از $\left(\frac{\sqrt{\alpha}-\sqrt{\beta}}{2}\right)$ شده ایم. و 
با دقت کامل گرئه بندی کرده ایم.