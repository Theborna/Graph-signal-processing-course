\documentclass[11pt]{article}
\usepackage[bordered]{uni-style}
\usepackage{wrapfig}
\usepackage{pgfplots}
\usepackage{circuitikz}
\usepackage{capt-of}
\title{آزمایشگاه تبدیل انرژی الکتریکی 1}
\prof{دکتر ذوالقدری}
\subtitle{مدار های مغناطیسی}
\subject{پیش گزارش آزمایش اول}
\department{مهندسی برق}
\info{
    \begin{tabular}{lr}
        برنا خدابنده & 400109898\\
    \end{tabular}
}
\date{\today}
\usepackage{xepersian}
\settextfont{Yas}
\begin{document}
\maketitlepage
\maketitlestart

\section{قسمت اول}

\textbf{
	1.
	در مدار مغناطیسی شکل زیر، اگر سیمپیچ با منبع جریان DC تغذیه شود؛ با فرض ناچیز بودن مقاومت سیم-
	پیچی و ثابت بودن شار هسته، رابطهی بین آمپر دور و شار را با فرض
	$l_g=0, l_g\neq 0$
	بدست آورید و با هم مقایسه
	کنید.
}

\begin{figure}[h]
	\centering
	\includegraphics*[width=0.3\linewidth]{result/1.png}
\end{figure}

کافیست مدار مغناطیسی معادل را در نظر بگیریم و معادلات سیستم را حل کنیم.

\begin{wrapfigure}{l}{0.4\textwidth}
	\centering
	\begin{circuitikz}
		\draw (0,0)
		to [short, i=$\Phi$] (4,0)
		to [R, l=$R_g$] (4,-3)
		to [R, l=$R_c$] (0,-3)
		to [voltage source, l=$\mathcal{F}(Ni)$] (0,0);
	\end{circuitikz}
	\caption{مدار مغناطیسی}
\end{wrapfigure}

\begin{align*}
	F   & = R\Phi = (R_c + R_g)\Phi = Ni                               \\
	R_c & = \frac{l_c}{\mu_r\mu_0 A}                                   \\
	R_g & = \frac{l_g}{\mu_0 A_g}                                      \\
	Ni  & = \frac{1}{\mu_0}(\frac{l_c}{\mu_r A}+\frac{l_g}{A_g})\Phi =
	\begin{cases}
		\frac{l_c}{\mu_r\mu_0 A}\Phi                             & l_g = 0   \\
		\frac{1}{\mu_0}(\frac{l_c}{\mu_r A}+\frac{l_g}{A_g})\Phi & l_g\neq 0
	\end{cases}
\end{align*}

به صورتی که $l_c$ طول موثر هسته است، $\mu_r$ ضریب گذردهی نسبی، $A$ مساحت موثر هسته،
$A_g$ مساحت موثر گپ هوایی با در نظر گرفتن اثراتی مانند fringing و $l_g$ فاصله هوایی است.


میتوان دید که با افزایش فاصله هوایی، شار گذرنده به شدت کاهش می یابد.

\clearpage

\section{قسمت دوم}

\textbf{
	2.
	اگر هسته فرومغناطیسی فوق دارای حلقه H-B به شکل زیر باشد، وجود فاصله هوایی چه تاثیری در مشخصه
	مغناطیسشوندگی از دید سیمپیچی دارد؟
}

\begin{figure}[h]
	\centering
	\includegraphics*[width=0.2\linewidth]{result/2.png}
	\caption{ مشخصه مغناطیس شوندگی هسته فرومغناطیسی}
\end{figure}

از دید تئوری معادلات زیر را داریم:

\vspace*{-1em}
\begin{gather*}
	Ni = H_cl_c+H_gl_g = H_c l_c + \frac{1}{\mu_0}\overbrace{B_g l_g}^{\frac{\Phi l_g}{A_g}} = 
	H_c l_c + \frac{1}{\mu_0}\frac{Al_g}{A_g}B_c\\
	Ni = H_{eff}l_c\Rightarrow H_{eff} = H_c(B_c) + \frac{Al_g}{\mu_0A_gl_c}B_c = H_{eff}
\end{gather*}

همینطور که از معادلات واضح است، وجود گپ هوایی باعث میشود در چگالی شار ثابت، میدان معناطیسی بیشتری داشته باشیم، 

در نتیجه انگار بطور افقی کشیده تر میشود مشخصه مغناطیسشوندگی معادل از دید سیم پیچ، و موجب کاهش 
شیب هیسترزیس میشود.

\section{قسمت سوم}

\textbf{
	3.
	 اگر سیمپیچی شکل 1 با ولتاژ سینوسی تغذیه شود، انتظار دارید شکل موج جریان عبوری از سیمپیچی چه
	شکلی داشته باشد؟
}

اگر اثرات غیرخطی سیستم را در نظر نگیریم، معادلات به شکل زیر هستند: 

\vspace*{-1em}
\begin{align*}
	V&=V_m\cos(\omega t)=-N\frac{d\Phi}{dt}\Rightarrow \Phi = \frac{V_m}{N\omega}\sin(\omega t)=\frac{F}{R}=\frac{Ni}{R}\\
	i&=\frac{V_m}{N^2\omega}R\sin(\omega t)=\frac{V_m}{N^2\omega}\left(\frac{1}{\mu_0}(\frac{l_c}{\mu_r A}+\frac{l_g}{A_g})\right)\sin(\omega t)
\end{align*}

پس جریان نیز یک جریان سینوسی ولی با اختلاف فاز 90 درجه خواهد بود، اثرات غیرخطی مانند هیسترزیس موجب مقداری اعوجاج 
در جریان شده در نتیجه این جریان دارای هارمونیک های بالاتر نیز خواهد بود در حالت کلی.
\footnote{عکس از تمرین سری دوم بدست آمده است}

\begin{figure}[h!]
	\centering
	\includegraphics*[width=0.8\linewidth]{result/res2.png}
\end{figure}

% \makeendpage
\end{document}

